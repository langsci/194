\chapter{Introduction}
The notion of \textit{meaning} has always been at the core of translation as a task as well as of translation studies (TS) as a discipline. As a task, translation is considered as an act of “communicating the overall meaning of a stretch of language” \citep[10]{baker_other_1992}. Within the discipline of TS, meaning is an essential concept of the metalanguage of translation and plays, with equivalence – with which it is so closely intertwined – a central role in translation theory \citep{halverson_concept_1997}.

If, as a task, translation is considered as an act of communicating meaning, as the above definition by Baker suggests, this seems to imply that the essence of the task lies within the \textit{transfer} of that overall meaning. The idea that translation is an act of transfer is furthermore suggested by the etymology of the English word \textit{translation}, which means `to carry across'. It is at this point that meaning becomes what I would call \textsc{the invariant of translation}. Meaning is what is transferred, it is the carefully wrapped content of a box labelled \textit{fragile} that at all times needs to be held securely, carried by a vigilant translator-delivery boy or girl. When the box is opened upon delivery, the deliverer’s mission will only be considered successful if the content of the box, once unwrapped, appears to be in the exact same state as when it was wrapped and dispatched by the sender. Any alteration to the box’s content is inconceivable, any broken glass or faded colors will necessarily be charged to the deliverer and the box will be returned to sender: invariant content (meaning) is the conditio sine qua non for delivery (translation). Many of the metaphors that are used to talk about translation, such as the one invoked here, adopt the idea of transfer (delivery), of the packing, unpacking and repacking (the box) of a message and its meaning (the content of the box). What is \textit{not} put into question whenever these metaphors are used, is that the content of the box needs to remain unaltered, in other words, that meaning presupposedly remains invariant; if not, delivery (translation) will not take place. But in times of Amazon and DHL, most of us would sense that, when opening the metaphorical box that was just delivered, we at times have this gut feeling that the long journey, the bumpy ride through rain and harsh weather may have somewhat impacted not only the box itself (the form) – as a logical consequence of the dispatch – but also the box’s content (the meaning), not so much that it is immediately apparent, but still. If we take the fragile-box-metaphor to the level of language, could it then be that the meaning of a word, so many times transferred from sender (a source language) to receiver (target language) by so many deliverers (translators), becomes somewhat altered upon reception in the target language, where the entirety of delivered goods constitutes the pile named \textit{translated} \textit{language}? Is it possible that the meaning of a word, in translated language, is or becomes (slightly) different from its meaning in non-translated language? And how can we investigate something as ephemeral as word meaning in translation? These are exactly the types of questions that the methodology proposed in this book aims to tackle.

Although many scholars explicitly or implicitly accept the idea that meaning is the invariant of translation,\footnote{The acceptance of meaning as the invariant of translation by a wide range of TS scholars is apparent from definitions of the concept \textit{tertium} \textit{comparationis}. This so-called ``third comparator'' is based on the idea “that an invariant meaning exists” \citep[31]{HatimMunday2004}, independent of both the source and the target text, and that it “can be used to gauge or assist transfer of meaning between ST and TT” \citep[31]{HatimMunday2004}.} it is however not a generally accepted given in TS. On the contrary, the idea that meaning is \textit{not} stable has generated a large body of research within socio-cultural studies of translation  \citep{baumgarten_ideology_2012}. This postmodernist view on meaning dismissed the linguistic view on meaning in TS, and in this way, the debate shifted away from the linguistic, “stable meaning” views in TS to a deconstructed, unstable view on meaning, embedded in cultural studies. In recent years, linguistically-oriented studies in TS have again come to the fore, but the status quo of meaning as the “invariant of translation” seems to be maintained. The aim of this book is therefore to investigate, from a linguistic viewpoint, meaning (un)stability in translation. Admittedly, the empirical investigation of meaning is not a straightforward endeavor, but in neighboring disciplines to TS such as lexical semantics, methodological solutions have been proposed. The development of a methodological solution to compare variance in meaning between translated and non-translated texts is one of the main objectives of this book. It will be illustrated by the investigation of the semantic relations of lexemes in the semantic field of inchoativity in Dutch, leading to a comparison of the semantic field of inchoativity in non-translated Dutch (SourceDutch) to the semantic field of inchoativity in translated Dutch (TransDutch). If meaning is indeed stable, semantic fields of translated and non-translated language should be identical. If however, meaning is not completely stable in translation, differences between the semantic fields are to be expected. 

Apart from being the content of the box in a translation task, meaning is also a metalinguistic concept in translation theory, where it is probably as pervasive as that of equivalence, although that does not mean that there is a consensus about the meaning of \textit{meaning.} In fact, metalinguistic discussions about the discipline’s core elements such as meaning and equivalence are more difficult in Translation Studies than in other disciplines due to the nature of the discipline itself: 

\begin{quote}
Above and beyond that the very nature of the discipline [Translation Studies] means that the discourse is conducted in and through a number of different languages, and with language being both the object of discussion and the means of communication, the risk of non-communication is only increased \citep[314]{snell-hornby_whats_2007}.
\end{quote}

As if to avoid venturing onto thin ice, recent theoretical paradigms in TS such as the universals\footnote{Universals are “features which typically occur in translated text rather than original utterances and which are not the result of interference from specific linguistic systems” \citep[243]{baker_corpus_1993}.} paradigm, seem to circumvent the whole idea of meaning, implicitly considering it as an integrative part of what translation is, rather than engaging in what seem to be an endless theoretical discussion. Numerous corpus-based studies within this paradigm \citep{poyatos_punctuation_1997,laviosa_core_1998, laviosa_corpus-based_2002,olohan_strange_2000,olohan_reporting_2000,baker_corpus-based_2004,bernardini_practice_2011,delaere_is_2012,oakes_lexical_2012,Kruger2012} have focused on lexical and grammatical phenomena (the packaging of the box) and have somewhat neglected the semantic level \citep[28]{laviosa_corpus-based_2002} (the content of the box).\footnote{This does not mean that the role of semantics itself in translation has not been addressed \citep{lewandowska-tomasczyk_specification_2010}, but this kind of research is rarely corpus-based and barely ever involves denotational issues.} To my knowledge, the question whether (universal) tendencies of explicitation, simplification, normalization or levelling out can be found on the semantic level has not yet been raised in TS.

With this work, I want to answer the three questions that arise here about meaning in translation:

\begin{itemize}
\item 
How can we investigate semantic differences in translated vs.\ non-trans\-lated language?
\item 
Are there any differences on the semantic level between translated and non-translated language? 
\item 
If there are differences on the semantic level, can we ascribe them to any of the (universal) tendencies of translation?
\end{itemize}

In order to answer these questions, I will first propose a methodological framework which offers a strategy to operationalize the idea of semantic difference between translated and non-translated texts. Secondly, the exploration of the semantic field of inchoativity in Dutch will enable me to tackle the second and the third question I aim to answer with this study. All this will finally lead to the formulation of a number of recommendations for future research about (universal) tendencies of translation on the semantic level.

The outline of this book is as follows. \chapref{sec:2} provides the theoretical foundation of this work. Every section of this chapter constitutes a building block necessary to arrive at the methodology presented in \chapref{sec:3}. In the first part of the theoretical chapter, I will zoom in on a number of aspects of corpus-based translation studies (CBTS) which form an integral part of this study: the \hyperlink{Corpora}{use of corpora in TS}, the \hyperlink{Bakersuniversals}{translation universals} and the cognitive turn in TS. I will equally discuss the place of the study of meaning within CBTS as well as the relationship between (the study of) universals, (the study of) meaning and the notion of equivalence. In the second part of this chapter I will look into different sub-disciplines of linguistics such as contrastive corpus linguistics and corpus semantics, which have, compared to CBTS, a much longer tradition of investigating meaning relationships. The theoretical foundations for the development of a bottom-up, statistical visualization method of semantic fields in both translated and non-translated language will be laid here. I will zoom in on the possibilities offered by the existing technique of semantic mirroring which uses the procedure of back-translation, the usefulness of statistical techniques for visualization purposes and the necessity of a theoretical framework within which the created visualizations can be interpreted.

\chapref{sec:3} contains a thorough description of the methodology. The method which is developed is an extension of an existing method, the Semantic Mirrors Method (SMM) \citep{johansson_translational_1998, aijmer_translations_2004, langemets_translations_2005}; it is corpus-based, uses statistical visualization techniques and consists of two parts (two extensions to the SMM). The first extension allows the potential user of the method to select candidate-lexemes for a semantic field. The second extension to the SMM proposes a way to visually inspect the retrieved data set(s). The ultimate goal of these extensions is to enable the user to compare visualizations of semantic fields of translated and non-translated language to each other.

In \chapref{sec:4}, I apply the methodology to the semantic field of inchoativity in Dutch. The choice of inchoativity as a ``test case'' is certainly not the most obvious choice, but offers a number of advantages: 

\begin{enumerate}
\item I expect to find high corpus frequencies of lexical items expressing inchoativity, which will facilitate statistical processing; 
\item for two central Dutch expressions of inchoativity, viz. \textit{beginnen} and \textit{starten}, close cognate translations are available in English (\textit{to} \textit{begin} and \textit{to} \textit{start}) but this is not the case in French (a particularity which can possibly offer interesting contrastive perspectives, e.g. about the impact of close cognates on the structure of semantic fields of translated language); 
\item the meaning differences between the expressions of inchoativity are expected to be (very) fine-grained \citep{schmid_introspection_1996}. 
\end{enumerate}

Inchoativity is therefore a compelling test case when one is interested in revealing meaning differences. The results are presented and described on the basis of three main visualizations, one for a semantic field of inchoativity in non-translated Dutch (SourceDutch), one for translated Dutch with English as a source language (TransDutch\textsubscript{ENG}) and one for translated Dutch with French as a source language (TransDutch\textsubscript{FR}). The goal is to explore the semantic field of inchoativity in Dutch and by doing so, to formulate an answer to the second and the third question of this study: are there any differences between translated and non-translated language on the semantic level, and, if there are, can we ascribe them to any of the (universal) tendencies of translation (we will focus on levelling out, normalization and shining through)?

In \chapref{sec:5}, an attempt will be made to connect the obtained results to current hypotheses in corpus-based cognitive translation studies and neurolinguistics. Two cognitive explanational hypotheses will be put forward and tentatively applied to the results of this study: the Gravitational Pull Hypothesis, developed by Sandra Halverson and the Neurolinguistic Theory of Bilingualism, by Michel Paradis.

\chapref{sec:6} concludes this book with an overview of the main findings with regard to the differences and similarities of the semantic relationships in translated Dutch compared to non-translated Dutch for the semantic field of \textit{beginnen}/inchoativity. In the concluding discussion, I will comment on the methodological contribution this work possibly makes to the empirical study of semantics in translation, especially with regard to the impact of translation on semantic representations. Finally, a number of recommendations for future research about (universal) tendencies of translation on the semantic level will be made. This book will then end where research into semantics in translation could begin, with the possibility of taking the conclusions drawn in this work as a starting point.
